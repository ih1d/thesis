% Author: Isaac H. Lopez Diaz <isaac.lopez@upr.edu>
% Description: Literature survey

\documentclass{article}

\title{Placeholder}

\author{Isaac H. Lopez Diaz}

\begin{document}
\maketitle

\section{Introduction}

This paper presents the design of a reflective programming language (PL) that reasons about its
dataflow semantics. A PL is said to be reflective when is able to reason about itself. \cite{smith}
It can be thought of as the process of converting data into a program. The inverse of this process,
reification, can be thought of turning a program into data. \cite{reification} These two processes 
allow a programmer to see the contents of the current execution, much like debugging. However, 
unlike debugging, one can change the semantics of the language on-the-fly. \cite{duplication}

The goal of the language is to better understand dataflow semantics. One such application would be on 
machine learning (ML) programs, since ML programs rely on dataflow graph execution models. \cite{tfmodel} 
The conversion from imperative to graph execution has proven to be challenging for programmers, 
primarily looking to optimize their code, leading to bugs or perfomance issues (the opposite
of what the programmer intended to do). \cite{imptograph} 

The argument is that by having a language to have programmable semantics that allows
programmers to specify the semantics (dataflow, procedural, imperative, etc.) leads to
less bugs. \cite{omlpl} 

\section{}

\section{}

\section{}

\bibliography{refs}
\bibliographystyle{plain}
\end{document}
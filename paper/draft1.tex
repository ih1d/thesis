\documentclass{article}

\title{A reflective language for the analysis of dataflow semantics}

\author{Isaac Hiram Lopez Diaz}

\begin{document}
\maketitle

\begin{abstract}
This paper presents the design and implementation of a reflective programming language able to 
reason about its dataflow semantics.
\end{abstract}

\section{Introduction}
A language (or any system, for that matter) is reflective when it is able to reason about itself.\cite{smith}
The result from this is that one would have a tower of interpreters, each interpreting the one "above it", meaning,
an interpreter interpreting an interpreter interpreting an interpreter, and so on. This would give the power of some
interpreter to "reach down" the tower and change the semantics of the interpreter interpreting it and extend the language. 
Friedman and Wand simplify this notion by introducing two processes: (1) \textit{reification},  the ability to transform 
code into data, and (2) \textit{reflection}, the ability to transform data into code. \cite{reification} Now instead of a
tower of interpreters one can have two interpreters 
\bibliography{refs}
\bibliographystyle{plain}
\end{document}